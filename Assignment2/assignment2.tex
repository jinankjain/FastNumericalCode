\documentclass[letterpaper, 11pt]{article}

\usepackage{amsmath, amsthm, latexsym, amssymb, graphicx, bold-extra, mathrsfs, frcursive}
\usepackage[pdftex]{color}
\usepackage[T1]{fontenc}
\usepackage{listings}

% Simplifies margin settings
\usepackage{geometry}
\geometry{margin=1in}

% Puts list item indicators in bold; makes flush with previous margin
\renewcommand\labelenumi{\bf\theenumi.}
\renewcommand\labelenumii{\bf\theenumii.}
% setlength\leftmargini{1.4em}
\setlength\leftmarginii{1.4em}

% Flexibility for headers and footers
\usepackage{fancyhdr}
\pagestyle{fancyplain}
\fancyhf{} %clear all header and footer fields
\lhead{\bf \small How To Write Fast Numerical Code \hspace*{\fill} Page \thepage}
\headsep 0.2in
\thispagestyle{empty}
\renewcommand{\headrulewidth}{0pt}
\renewcommand{\footrulewidth}{0pt}

\parindent 0in
\parskip 10pt
\setlength{\headheight}{20pt}

\title{ETH Zurich}

\begin{document}

%=======================================

\begin{center}
\Large \bf 263-2300-00: How To Write Fast Numerical Code

\Large \bf Assignment 2: 80 points

\large Submitted by Jinank Jain
\end{center}

\textbf{Solution 1}\\ \\
\bigskip

\textbf{Solution 2}\\ \\
\textbf{Part b:} \\
In this part I tried four different variations which are discussed as below:
\begin{itemize}
	\item In the first approach I just unroll the inner loop and does not do any modifications with that unrolled loop. The function is called \texttt{loop\_unroll} in C file.
	\item In the second approach, I did some optimization with unrolled loop that I did in the previous section, I replaced $x[i]$ with some some scalar $a$ which stores the value of $x[i]$ in some register. Another optimization was to much multiple accumulator instead single sum. I used for different sum variable to store the intermediate result. The function is called \texttt{scalar\_replacement} in C file.
	\item In the third approach I unrolled the first loop too. Since in the array y there are only 5 values and I want them to be stored in registers so I unrolled first loop in multiple of 5 and storing all the values in the register. And creating 20 different sum accumulator and summing up the final answer. The function is called \texttt{super\_scalar\_replacement} in C file.
	\item In the fourth approach I remove the assumption which I took in third approach that $n$ was divisible by 5 so for that I need introduce some branching instructions. The function is called \texttt{super\_scalar\_replacement\_generalize} in C file.
\end{itemize}

\textbf{Part d:} \\
\begin{table}[h!]
\centering
\label{my-label}
\begin{tabular}{|l|l|l|l|l|l|}
\hline
\textbf{\begin{tabular}[c]{@{}l@{}}Optimization\\ Flags\end{tabular}} & \textbf{\begin{tabular}[c]{@{}l@{}}slow\\ performance1\end{tabular}} & \textbf{loop\_unroll} & \textbf{\begin{tabular}[c]{@{}l@{}}scalar\\ \_replacement\end{tabular}} & \textbf{\begin{tabular}[c]{@{}l@{}}super\\ \_scalar\\ \_replacement\end{tabular}} & \textbf{\begin{tabular}[c]{@{}l@{}}super\_scalar\\ \_replacement\\ \_generalize\end{tabular}} \\ \hline
-O0                                                                   & 0.164 FLOPs/c                                                        & 0.238 FLOPs/c         & 0.233 FLOPs/c                                                           & 0.817 FLOPs/c                                                                     & 0.738 FLOPs/c                                                                                 \\ \hline
-O3                                                                   & 0.537 FLOPs/c                                                        & 0.549 FLOPs/c         & 0.544 FLOPs/c                                                           & 4.881 FLOPs/c                                                                     & 3.527 FLOPs/c                                                                                 \\ \hline
\end{tabular}
\caption{Performance comparison between different optimization function}
\end{table}

\bigskip

\textbf{Solution 3}\\ \\

\begin{table}[]
\centering
\caption{My caption}
\label{my-label}
\begin{tabular}{|c|c|c|c|c|c|c|c|c|c|c|c|c|}
\hline
\textbf{Instruction} & \multicolumn{3}{c|}{\textbf{Reference Values}} & \multicolumn{3}{c|}{\textbf{Regular Case}} & \multicolumn{3}{c|}{\textbf{\begin{tabular}[c]{@{}c@{}}Special Case:\\ Mul: 1\end{tabular}}} & \multicolumn{3}{c|}{\textbf{\begin{tabular}[c]{@{}c@{}}Special Case:\\ Div: 2\end{tabular}}} \\ \hline
 & \textbf{Latency} & \textbf{TPS} & \textbf{Gap} & \textbf{Latency} & \textbf{TPS} & \textbf{Gap} & \textbf{Latency} & \textbf{TPS} & \textbf{Gap} & \textbf{Latency} & \textbf{TPS} & \textbf{Gap} \\ \hline
\textbf{MULSS} & 5 & 1 & 1 & 5.34 & 0.76 & 1.32 &  &  &  &  &  &  \\ \hline
\textbf{MULSD} & 5 & 1 & 1 & 5.06 & 0.91 & 1.09 &  &  &  &  &  &  \\ \hline
\textbf{DIVSS} & 11-14 & 0.16 & 6 & 13.07 & 0.13 & 7.49 &  &  &  &  &  &  \\ \hline
\textbf{DIVSD} & 15-20 & 0.07 & 14 & 20.06 & 0.07 & 14.06 &  &  &  &  &  &  \\ \hline
\end{tabular}
\end{table}

\bigskip

\clearpage

%=======================================

\end{document}